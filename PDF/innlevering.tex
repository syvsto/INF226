\documentclass{report} %skal vi legge til [a4paper] og endre til report? 

\usepackage[a4paper]{geometry}
\usepackage[backend=bibtex, style=verbose-trad2]{biblatex}
\usepackage[utf8]{inputenc}
\usepackage{comment}
\bibliography{kilder}
\linespread{1.3}

\title{INF226 Obligatory assignment}
\date{2016-10-02}
\author{Håvar Eggereide and Syver Storm-Furru}

\begin{document}
\maketitle
\pagenumbering{gobble}
\newpage
\pagenumbering{arabic}
\tableofcontents
\newpage
\chapter{Introduction}

  For our obligatory assignment we were tasked with analysing OpenMRS, a medical
  patient journal system. The analysis is performed both using static and
  dynamic code analysis tools (HPE Fortify and FindBugs for static analysis and
  OWASP ZAP for dynamic), as well as a thorough manual run-through of the installation
  process and software usage.

  \paragraph{The goals of software testing}

  It is now common practice  for a developer to adhere to a software architect's
  plan for the development of the software components, and many utilizes
  unit-testing as a supporting tool for developing code without errors. So what
  is then the benefit of having a separate plan and procedure for system testing
  from the view of security?

  It has been demonstrated with many grim examples during this course and in the
  curriculum texts, the importance of security and the heavy consequences to the
  whole enterprise if it is neglected. The issue is of such a big scope that
  approaching it as just another design or implementation issue fails to
  appreciate the complexity it presents.

  The goal of testing the security as a separate process is to try to map as
  much as possible of the potential security issues and security considerations
  that has to be kept in mind walking thru the process of development. To
  accomplish this it is necessary to approach the software and the architecture
  from the point of view of a agent with malicious intent.

  In the end it is  most likely not possible to make the software completely
  immune to security faults. It's a natural part of security testing to find
  problems where a full repair might not be achievable, it is then also the
  responsibility of the security testing process to develop mitigation plans for
  these vulnerabilities.

  So the goal of security testing is to find, repair and prepare for security
  vulnerabilities.
  
  \paragraph{Static and dynamic analysis}

  Static and dynamic analysis are methods for checking the software for vulnerabilities.
  They are both done using software tools that help the user analyze. Static
  analysis is performed before any code is run, and checks the code base for
  vulnerable code that can be abused. It checks different runtime behaviours,
  both standard and non-standard, in order to find potential backdoors, weak
  code and other vulnerabilities. Dynamic analysis on the other hand is done
  while the application is running, and works a lot like how an actual attacker
  would target the application, performing such things as XSS and SQL incjection
  attacks, as well as monitoring the system runtime, for example performance and
  HTTP requests \autocite[]{VERACODE}.

\newpage
\section{Software involved}
  
\subsection{OpenMRS}

  OpenMRS is, according to the website, the ``world's leading open source
  enterprise electronic medical record system platform'' \autocite[]{OpenMRSAbout}. It is used in
  hospitals and medical facilities all over the world, for example in Nigeria, South Africa, India
  and the United States, and is supported by many different governments, NGOs,
  and both for- and non-profit organisations. The software has a stated goal of
  being usable with no programming knowledge, and to be a common platform for
  which medical informatics efforts in developing can be built.

  \paragraph{Technical Specifications}

  OpenMRS is a client-server platform, with a web front end. It is programmed in
  Java 7, using Tomcat 6 or 7 as the server framework, and MySQL 5.6 as the
  database backend. It also exposes a programming API to users, and is modular
  and extendable.

  

  \paragraph{Usage}

  OpenMRS is operated through a web browser, and is centered around a main dashboard that,
  with just the base installation, will give you access to patient journals, as well as
  showing currently active patients. The system also lets you categorize patients/other registered
  in different persons in different categories, easing the process of generating statical analyses.
  The system also lets your register and book future meetings with patients.

\subsection{HPE Fortify}

  Fortify is a code security tool suite, developed by Hewlett-Packard
  Enterprise (HPE). It aims to ``make application security a natural part of the new
  SDLC, enabling time to market by building security in''\autocite[]{Fortify}.
  It contains such tools as WebInspect, a dynamic code analysis tool, and the
  Fortify Static Code Analysis tool. Fortify is a proprietary solution, but is
  available with an academic license for free.

  We ran the Audit Workbench version bundled with Fortify version 16.10.

\paragraph{Audit Workbench}

  Audit Workbench is the tool used to organise the the output of HPE's static
  code analysis software contained in the Fortify package. It is a GUI
  application built on top of Eclipse, specially designed for organising and
  presenting output for the HPE tools. 
  
  The installation process for Audit Workbench was straight-forward, but running
  the program required changing variables for the Eclipse backend, without
  information about how this is done readily available. The software also was a
  large RAM consumer, needing 5 gigabytes of RAM to analyze a relatively large
  project (OpenMRS).

\subsection{FindBugs}

FindBugs is a static analysis tool for scanning java code looking for potential errors in the
implementation. It is distributed under Lesser GNU Public License. The project
originated from the University of Maryland.\autocite[]{FindBugs} It runs on java
1.7 ether initiated from a terminal or you may use a simple GUI which makes it
easier to get familiar with the program. 

We ran FindBugs 3.0.1 for our tests.
 

\subsection{OWASP ZAP}

ZAP is a dynamic analysis tool developed by OWASP, the Open Web Security
Project. The software acts as a proxy between the host computer and a web
application, performing different types of automatic scans, as well as having
tools for manual searches for security vulnerabilities\autocite[]{ZAP}.

We ran ZAP version 2.5.0 for our tests.

\subsection{ThreadSafe} 

``ThreadSafe is a static analysis tool for finding concurrency bugs and
potential performance issues in Java programs.''
 \autocite[]{THREADSAFE}

The program is developed by Contemplate ltd. aiming to help developers discover
concurrency issues in the code-base prior to the deploying of the application.
Concurrency is a complicated issue, having the correct understanding of
what it is and how it works is not easy. Then the next step
of implementing it correctly can give rise to more complicated issues which are
even harder to discover. For the developer to have a tool to aid in working with
concurrency is of high value. Should a issue arise during runtime the
ramifications, for a system like OpenMRS which potentially need to be operating
24 hours a day, could possibly be disruption of the integrity of the database
and halt of operations.

ThreadSafe is not open source and it is necessary to invest in a license if
there is a need to integrate the application in the SDLC. There is however a
free trial of 14 days available.
%  LocalWords:  HPE

\chapter{Setup and Configuration}
\subsection{OpenMRS}
  The setup process of OpenMRS is quite involved and time consuming when
  attempting to do so on a personal computer, requiring both Tomcat, Java and
  MySQL to be setup. The official documentation is useful, but different parts
  of it uses different versions of e.g. Tomcat, so it can be confusing.
  It also provides install instructions for Windows and Linux distributions
  with Aptitude, but not for OSX or other Linux distributions. 

  As mentioned, OpenMRS doesn't run on the newest version of MySQL (at the time
  of writing MySQL 5.7), and the install instructions do not mention this. The
  process of figuring this out, and of removing and reinstalling a previous
  version of MySQL, proved to be a lengthy detour on an already long road.
  The instructions are also not very specific when noting which files you need
  to run OpenMRS in Tomcat, whether it is the source code, which was difficult
  to build and only return a test suite on a normal compile, a readily packaged
  complete install (which did not work properly), or a .war file that should be
  uploaded to the Tomcat server.

  We first attempted setting up OpenMRS on OSX 10.11, but ran into problems when
  trying to install the correct version of MySQL, and therefore retried in an
  empty virtual machine running Linux (tested with both Lubuntu and Kali Linux).
  Following the install instructions were a lot easier when running Debian based
  distributions containing Aptitude, but we still had to find and install a
  previous version of MySQL. 

  Once everything was installed OpenMRS had some extra setup that was required,
  done through a web interface. This was mostly easy once the correct version of
  MySQL was in place. You were also given a first username and password that
  was, respectively, `admin' and `Admin123'.


\subsection{Tools}
\paragraph{Audit Workbench}

Audit Workbench was set up by downloading the HPE Fortify software suite, which
contains installers for different Fortify software. The installer for Audit
Workbench was run, without any issues. Setting up the Audit Workbench IDE
required changing the -Xmx setting in a deeply nested ``eclipse.ini'' file,
which required a bit of dedication to find. The reason for this is that scanning
a project requires more memory than is allocated to Audit Workbench by default.
After that, running Audit Workbench was pretty straight forward.

\paragraph{FindBugs}

The stand-alone application which is distributed from the projects main page was
easy to get started from the terminal(OSX-bash) without any setup other than the
first installation. Starting the initial scan was also straight forward. The
high number of errors reported was misleading and proved a cumbersome task. What
proved to be a better solution was using a FindBug-plugin for the IntelliJ IDEA.
This way IntelliJ made a build of the code base and then ran the scan on this
build. IntelliJ could reason about the classes which where not necessary
to scan, as per example all the unit tests. The code base we used for the scan was
cloned directly from github (https://github.com/openmrs/openmrs-core).

\paragraph{OWASP Zap}

Running ZAP requires ZAP to be installed in the same location as OpenMRS when
running OpenMRS locally. Therefore, we set up a virtual machine to run both ZAP,
the Tomcat server and the MySQL server. The virtual machine used ran the the
linux distribution Kali Linux, which is a specialized distro for penetration
testing. It included OWASP ZAP, and we therefore did not need to set up the
application itself. However, fully utilizing ZAP required using an addon for our
browser (Firefox). The addon had an out-of-date signature, and thus needed
Firefox to run addons without valid signatures. After changing that setting for
Firefox and running the setup wizard for ZAP inside Firefox, it was ready to be used.

\paragraph{ThreadSafe} % Setup and Configuration

ThreadSafe has three install options, two which are available in the trial
version. These options are: A .jar file which is executable from the terminal
taking input-arguments and a setup file for each project, plugin for a static
analyzing tool called SonarQube and a plugin for the Eclipse IDE.

As we were using a free trial we had to be limited to ether using the terminal
program or the Eclipse plugin, which were the included options in the trial. As
setting up a project configuration file for the terminal application seemed to
open the possibility of misconfiguration, using the Eclipse plugin seemed the
best alternative. Installing the plugin from the .zip file was easy, the only
other thing necessary was to order a trial license and paste this license in the
Preferences subcategory ThreadSafe. We then let Eclipse build the Maven .pom file from
the OpenMRS source-code and everything was ready for scanning. 

\section{Preliminary results}

\subsection{Visible potential issues}
  During the install process we were on the lookout for potential vulnerabilities that
  were visible to us. Two things stood out:
  1) The program was set up with a static username and password (resepctively `admin'
  and `Admin123'), and we were never prompted to change this. This means that if the system
  installer doesn't notice this is the case, the is a user with administrator priviledges
  available to anyone, which is pretty bad.
  2) The system didn't work with the newest version of MySQL. The latest version it can run (newest 5.6 version)
  has several vulnerabilities, some very serious \autocite[]{CVEDetails}. While some of these
  also are present in newer versions of MySQL, others are not, making the database slightly more vulnerable.

\chapter{Tests of the software tools}

  After we set everything up, we ran a test run of each analysis tool, to make sure
  everything was functioning properly and to find any glaring issues, if any.

\paragraph{Dynamic analysis of the OpenMRS Web GUI}
  
  The initial Quick Start scan of our test setup of the OpenMRS service only had
  the login front page to crawl. This yielded very few results considering this
  is just one page. It found some information about jQuery, ZAP also warned
  about the implementation of the cookie.
  
\paragraph{Audit Workbench report}  

  The first scan yielded a few messages of high concern. The OpenMRS core
  includes unit-test which gives quite a few erroneous warnings concerning
  security risks and more general bad coding practices. 

  There was an error concerning how the cookie was coded which may be
  interesting to look more into.

\paragraph{FindBugs code scan}

  Running the scan seems easy and most of the work is obviously analysing the
  results. The reports might be a bit hard to navigate and there was some
  trouble generating a html report that potentially is more readable. Again the
  unit-test generated false positives.

\paragraph{ThreadSafe} % Test of the software tools

The initial scan gave a few possible errors, all of them concerning either the
OpenMRS-API or the OpenMRS-WEB which might make sense as these expose surface to
external modules. The scan also gave some false positives referring to
unit-test.
  
%  LocalWords:  SDLC SSL
  \newpage

\section{Findings}

\subsection{Raw data}

\paragraph{Fortify}

  Audit Workbench found 60 critical vulnerabilities, as well as 115 issues
  considered highly vulnerable. Following is a list of the categories designated
  by Audit Workbench for these vulnerabilities, as well as the number of
  occurrences of each vulnerability and a short explanation.

  The critical vulnerabilities were in the
  following categories:
  
  \begin{itemize}
  \item Command injections (4): The program tries to call an external command
    without validating where the command is from.
  \item Insecure SSL (3): Uses SSL certificates are based on default system
    Certificate Authorities, which may be compromised, and sometimes are.
  \item Hard coded passwords (3): Lets an attacker log in if he or she has access
    to the source code, which in this case means anyone.
  \item Path manipulation (2): The program accesses a file with a file path that
    can be chosen by the attacker.
  \item Privacy violation (49): The program mishandles confidential information
    in some way or another, for example by writing it to disk.
  \end{itemize}

  The vulnerabilities rated high are as follows:

  \begin{itemize}
  \item Header manipulation with cookies (1): The program doesn't validate data
    in an HTTP request, so cookies can be tampered or spoofed.
  \item Hard coded encryption key (3): Same issue as with hard coded passwords.
  \item Log forging (12): The program doesn't validate inputs that get written
    to the log, meaning malicious data could be entered.
  \item Empty passwords (6): The program assigns empty passwords, making it much
    easier to break into the application.
  \item Hard coded passwords (25): Same issue as with the critical hard coded passwords.
  \item Passwords in Configuration File (36): Passwords are stored in plain-text
    in a configuration file, meaning an attacker can get the password by getting
    the file.
  \item Privacy violation (10): Same issue as in the critical privacy violations.
  \item Privacy violation from heap inspection (3): The program stores
    confidential information in strings, which are immutable and thus stored in
    memory until the JVM garbage collector is run (which may not occur for a
    while). 
  \item Race Conditions from Singleton member field (19): The program uses a
    singleton class which takes concurrent connections, meaning data can be
    shared between users.
  \end{itemize}

\paragraph{FindBugs}

  The first scan detected 1058 potential bugs. Some of them are from test classes

%  LocalWords:  JVM
  and will not be mentioned here. The report generated from the FindBugs GUI was
  very uninformative and hard to navigate. We found a plugin for IntelliJ IDEA
  that used Findbugs as the back-end and produced a more informative report
  where it was easy to exclude the testbase as IntelliJ had generated a build of the
  codebase giving a sense of which code was where. In this new report we
  had a total of 296 warnings and a separation between what FindBugs was very
  certain was a bug and what maybe could be a bug. In the following text we have
  made a presentation of the bugs FindBugs is most certain is a potential
  problem.


Correctness
  \begin{itemize}
    \item Masked Field:

    Class defines field that masks field in superclass
    org.openmrs.web.filter.StartupFilter:
    This error is mostly harmless. It may have potential for generating
    errors in later updates to the code. The field is marked as
    protected in the superclass so it is excessive in the child and
    probably should be removed.

    \item Masked Field:

    Field UpdateFilter.log masks field in superclass
    org.openmrs.web.filter.StartupFilter:
    Same as above.
  \end{itemize}
Bad practice
  \begin{itemize}
    \item Dropped or ignored exception:

      This method might ignore an exception. In general, exceptions
      should be handled or reported in some way, or they should be thrown
      out of the method.

      This error occurred in the following classes:

      \begin{itemize}
        \item ServiceContext.setModuleService
        \item WorkflowCollectionEditor.setAsText
        \item OpenmrsUtil.parse
        \item BooleanConceptChangeSet.getInt
        \item BooleanConceptChangeSet.createConcept
      \end{itemize}
    \item Initialization circularity. 

      Class makes active use of subclass during initialization and might
      result in null value.

      Class:

      \begin{itemize}
        \item Operator 
      \end{itemize}
  \end{itemize}
Internationalization 
  \begin{itemize}
    \item Reliance on default encoding. 
      Found code that does a bit to string conversion while assuming
      the default platform encoding is suitable. This could make the code
      behave differently on different platforms.

      Classes:

      \begin{itemize}
        \item HL7ServiceImpl
        \item ModuleFileParser
        \item ModuleUtil
        \item TextHandler
        \item DatabaseUpdater
        \item HTTPClient
        \item OpenmrsUtil
        \item CreateCodedOrderFrequencyForDrugOrderFrequencyChangeset
        \item MigrateConceptReferenceTermChangeSet
        \item SourceMySqldiffFile
        \item Listener
        \item StartupFilter
        \item TestInstallUtil
      \end{itemize}
  \end{itemize}
Malicious code vulnerability
  \begin{itemize}
    \item Mutable static field. 
      Public static field isn't final and could be change by malicious
      code or by accident.

      Classes:

      \begin{itemize}
        \item SchedulerConstants
        \item OpenmrsConstants
        \item Security
      \end{itemize}
  \item Field is mutable array. 
    Same as above and field is referring to an array.

    Class:

    \begin{itemize}
      \item OpenmrsConstants
    \end{itemize}
  \end{itemize}
Multithreaded code vulnerability
  \begin{itemize}
    \item Unintended contention or possible deadlock due to locking on shared
      objects: 
      Synchronization on Boolean. There is a risk of syncing on a object
      belonging to another process since there is only two Boolean objects.
      Resulting in a deadlock

      Class:

      \begin{itemize}
        \item HL7InQueueProcessor
      \end{itemize}
    \item Unsynchronized Lazy Initialization: 
      Incorrect lazy initialization and update of static field. Static field
      is not completed and could be accessed by another process while it is
      still initializing which would lead to a multithreading bug.

      Class:
      \begin{itemize}
          \item OpenmrsClassLoader
      \end{itemize}
  \end{itemize}
Performance
  \begin{itemize}
    \item Questionable Boxing of primitive value. 
      Boxing a primitive to compare. It's more efficient to use static compare
      method on primitives.

      Class:
      \begin{itemize}
          \item ModuleFactory
      \end{itemize}
    \item Boxing/unboxing to parse a primitive. 
      A sting is boxed just to be unboxed for parsing. More
      efficient to use static parse*** method.

      Class:
      \begin{itemize}
        \item BaseHyphenatedIdentifierValidator
      \end{itemize}
  \end{itemize}
Security
  \begin{itemize}
    \item Potential SQL Problem. 
      Nonconstant string passed to execute or addBatch method on an SQL
      statement The method invokes the execute or addBatch method on an SQL
      statement with a String that seems to be dynamically generated.
      Consider using a prepared statement instead. It is more efficient
      and less vulnerable to SQL injection attacks. 

      Classes:
      \begin{itemize}
        \item ConceptValidatorChangeSet
        \item MigrateAllergiesChangeSet
      \end{itemize}
  \end{itemize}
Dodgy code
  \begin{itemize}
    \item Dead local store. This instruction assigns a value to a local variable, but the
      value is not read or used in any subsequent instruction. Often, this
      indicates an error, because the value computed is never used.
      Note that Sun's javac compiler often generates dead stores for final
      local variables. Because FindBugs is a bytecode-based tool, there is
      no easy way to eliminate these false positives.

      Class:
      \begin{itemize}

%  LocalWords:  multithreading
        \item PatientServiceImpl
      \end{itemize}
  \end{itemize}
\paragraph{OWASP ZAP}

A passive run of ZAP returns the following four possible vulnerabilities:

\begin{itemize}
 \item X-Frame-Options header is not included in the HTTP response to protect against
  'ClickJacking' attacks: Considered a medium vulnerability, this
  vulnerability can let unknown sites be loaded inside frames or iframes.
 \item A cookie has been set without the HttpOnly flag: A low vulnerability that
   makes it possible to access the cookie using Javascript. 
 \item Web browser XSS protection not enabled: A low vulnerability where the
   HTTP header X-XSS-Protection isn't set, so that the browser doesn't use it's
   own cross site scripting protection.
 \item X-Content-Type-Options Header missing: Some older browsers try to
   automatically figure out the content type if X-Content-Type-Options is not
   set to 'nosniff', making it possible to do such operations as cross-site
   scripting \autocite[]{XFrame}.
\end{itemize}

When re-running ZAP and giving it access to an administrator user, as well as
manually performing the scans, another couple of vulnerabilities show up. The
new list contains the following, in addition to the previously mentioned:

\begin{itemize}
  \item SQL Injection: Considered a high vulnerability, it can grant access to
    an attacker through manipulation of SQL queries in input forms.
  \item Format String error: Considered a medium vulnerability, it happens when
    an input string is evaluated as a command.
  \item Incomplete or No cache-control and Pragma HTTP Header Set: Considered a
    low vulnerability, can let both the browser and proxies cache content
  \item Cross-domain JavaScript source file inclusion: Considered a low
    vulnerability. Is reported because ZAP has found a site using a script from
    another party.
\end{itemize}

\paragraph{ThreadSafe} % Raw data

Excluding false positives from tests the scan gave 12 major and 1 minor
warnings. No Critical or ``Blocker'' warnings were generated. The major warnings
are as follows:

\begin{itemize}
\item Inconsistent synchronization of access to Field/a collection.

  StartupFilter, UpdateFilter.

\item Mixed synchronization of access to a field.

  OpenmrsClassLoader, OrderSetServiceImpl, UpdateFilter.
  
\item No lock held while iterating on a synchronized collection view.
  
  UpdateFilter.

\item Non atomic use of Get/Check/Put.

  WebModuleUtil.

\item Synchronizing on reusable objects.

  ModuleFactory, WebModuleUtil.

\item Unsafe iteration over synchronized collection.

  HL7InQueueProcessor.
  
\item Unsynchronized access to field from asynchronously invoked method.
  
  WebModuleUtil.

\end{itemize}

\subsection{Interpretations of the raw data}

\paragraph{Tests}
Some of the critical errors found by both Fortify and FindBugs were in test
classes, and are therefore not going to run while the application is deployed.
These bugs and vulnerabilities are therefore disregarded. 

\paragraph{External files}
The vulnerabilities concerning passwords in clear text found by Fortify are
mostly false positives, as they are found in localization files, and don't
contain real passwords (instead containing the translation of the noun
password). We will therefore disregard any vulnerability found in localization
files. However, some files containing clear text passwords are used for setting
up the server, and are thus relevant.

\paragraph{Privacy violations}
Many vulnerabilities found by Fortify were in the category Privacy Violations.
Considering the type of application OpenMRS is (it manages patient journals),
these are more important than for other applications, as patient journals
contain lots of personal and sensitive information such as (possibly) social security
numbers and medications used. Many of these files are about the the application
logging login information and search terms, and the security of the logger is therefore of
particular interest.

The other privacy violation needs the attacker to access the JVM memory, and
therefore requires the attacker to have root access. Since the program runs on a
server, if the attacker has root access you've probably lost your database and
similar serious problems, and thus problems requiring memory access aren't as
important. This might be more of a problem if the server is external (e.g.
hosted on Amazon AWS or CloudFlare), but then, it is up to Amazon or CloudFlare

%  LocalWords:  Javascript ClickJacking
to secure their servers.

\paragraph{Misleading results}
Especially in the ZAP scan, multiple results, such as some (but not all) of the
vulnerabilities enabling ClickJacking attacks, are found in websites related to
the browser used and in the ZAP addon for Firefox. We will therefore disregard
all reported ZAP vulnerabilities that contain no mention of OpenMRS.

\paragraph{Race conditions}
Fortify found indications of problems relating to the use of singleton. This can
cause resources to be shared between users. This is both a privacy problem and a
possible corruption of the integrity of data fields.\autocite[]{owaspracecondition}

ThreadSafe also pointed to some inconsistencies in handling fields and
collocations of data. Most interesting is the two classes ModuleFactory and
WebModuleUtil which seems to be operating somewhere between the OpenMRS core and
possible external modules.
%  LocalWords:  iframes ModuleFactory WebModuleUtil ThreadSafe
%  LocalWords:  HttpOnly

\paragraph{Dynamic analysis}
The run of OWASP ZAP didn't find any critical or highly vulnerable issues,

%  LocalWords:  nosniff
indicating that the front end is relatively well secured. The ZAP issues were
related to HTTP headers (for example no SQL injection issues and direct XSS
issues). Since the critical issues were found by static analysis, we will focus
more on the server errors.

\subsection{Vulnerable external libraries/software}
In order to find vulnerabilities in libraries and software OpenMRS is based on
we looked at exploit-db.com, which is a database looking up vulnerabilities in
the CVE and OSVDB databases. We looked for exploits in Tomcat, JRE, Java, and
MySQL. We found the following relevant exploits:

\begin{itemize}
\item Tomcat: There are issues with the newest version of Tomcat (Tomcat 8) for
  OpenMRS, and we therefore looked for exploits in Tomcat 7. There are
  especially two exploits worth noting:
  \begin{itemize}
    \item EDB-ID 35011: Apache Tomcat 7.0.4 - 'sort' and 'orderBy' Parameters Cross-Site Scripting:
      Since Tomcat doesn't sanitize user inputs properly, cross-site scripting
      can be performed.
    \item EDB-ID 40450: Apache Tomcat 8/7/6 - Privilege escalation: The default
      Tomcat distributions available on Aptitude (Debian and Debian based distro
      package manager) as well as RPM (RedHat and RedHat derivate distro packagr
      manager) have a systemd config file that is insecure, run on boot, and can give
      root access to an attacker. The payload for this file can be delivered
      through a Tomcat shell in java web apps, or locally. This is a new exploit
      (found 10/10/2016) and thus unlikely to be patched on servers running OpenMRS.
  \end{itemize}
\item MySQL: EDB-ID 40678: MySQL/MariaDB/PerconaDB System user Privilege
  escalation / Race condition:
  Lets a local user access and run database operations above their own privilege
  level, and combined with other exploits, can give server root access to a user.
  
\item Java/JRE: OpenMRS runs on JRE 1.7, for which we couldn't find any relevant exploits.
\end{itemize}\autocite[]{exploitdb}

%  LocalWords:  XSS OWASP Servlet CloudFlare OpenMRS FindBugs
\chapter{Analysis, evaluation and recommendations}  
\section{Evaluation of tool reports}
\paragraph{Audit Workbench}
Audit Workbench presented us with 60 critical errors, of which nearly all were
found in tests. However, the vulnerabilities related to SSL certificates are not
related to tests, and would let an attacker spoof a certificate and insert
itself between the server and the endpoint. OpenMRS would trust the spoofed
certificate and therefore let the user recieve unauthorized data. This is
critical for instances of OpenMRS connected to the internet.

There are also several issues in the ``high'' category: For example hard coded
encryption keys, meaning anyone with access to the source code (which in an open
source project is anyone) has access to the encryption key. Since the project
uses AES encryption by default, which uses the same key for encryption and
decryption, it means anyone has access to the decryption key. Another issue is
that some default passwords are stored in cleartext, and while the application
asks for the system administator to change default passwords, the task is
delegated to the user, which may or may not comply. It would have been safer not
to store any passwords in cleartext. 

\paragraph{FindBugs}
Most of the FindBugs errors were in the category Malicious Code Vulnerablity
warnings, which, along with Security Warnings, is the most relevant category
for our purposes. The warnings given a high priority by FindBugs are warnings
about non-final properties that should be final. According to the description
from FindBugs, these fields could be changed by malicious code or by accident by
programmers working on OpenMRS or an OpenMRS module. Among the properties not
set as final are properties setting the data directory OpenMRS uses, as well as
a regular expression and other password requirement settings. If these get
changed by accident or on purpose, it could let an attacker use either point the
software to a location on the computer under his control, or by setting a custom
regex, whitelist characters that would let the attacker perform SQL injection attacks.

FindBugs also presents many warnings about OpenMRS returning references to
mutable objects, something that can let an attacker understand how OpenMRS
represents internal state and data. Again, this is mostly relevant if the
attacker writes a module for OpenMRS that gets installed and for example leaks
information about the running state of the application to the attacker. FindBugs
recommends returning a copy of data instead of mutating data in most cases. 

\paragraph{OWASP ZAP}
ZAP reported an SQL injection which was found in the Forgotten Password page of
OpenMRS. As this page is available to all persons, both users and non-users, it
is a quite significant vulnerability which can grant an attacker access to the
user database. ZAP also reported multiple examples of ClickJacking
vulnerabilities. Some of them are not in OpenMRS itself, but rather in a plugin
for Firefox used by ZAP, and can be disregarded. Others are, however, in
OpenMRS. Depending on the functionality in OpenMRS added by modules ClickJacking
can be an issue, as it requires overlaying some sort of CSS and/or Javascript to
trick users into clicking on something they did not intend to click. Some kind
of XSS attack would therefore be needed. If for example a module is added to
OpenMRS that recieves emails form patients and lets a doctor answer them, an attacker
could perform a ClickJacking attack by overlaying an iFrame in the sent message
and sending it to the doctor. ZAP recommends setting X-Frame-Options to for
example DENY. OpenMRS has not set this header by default.

\paragraph{ThreadSafe}
As Fortify flagged a possible problem with the implementation of the singleton
in openMRS with the possibility of information being shared between users we
felt the need to dig deeper in the possibilities of problems in the concurrency.
We introduced ThreadSafe as a dedicated tool to statically analyze
the codebase, as concurrency is a complicated issue we thought a tool that
was dedicated to searching for problems with concurrency would give results that
could be trusted more.

ThreadSafe pointed to inconsistencies in accessing fields and lists in several
classes. There seems to be inconsistent locking of the libcachefolder in the
openmrsClassLoader. Most interesting seems to be the number of inconsistencies
in locking in the ModuleFactory, WebModuleUtil and in some of the filter
classes. There are many classes in OpenMRS that uses the java.util.concurrent library to
handle locking of resources, tho none of them seem to be a part of the OpenMRS
web library. As modules are a key element in the operation of OpenMRS is seems
critical that a collection of classes that relate to each other, as the
ModuleFactory most likely is related to the WebModuleUtil and WebModuleUtil uses
filters and initiates singletons, all seems to contain the same possible error
types and these types are crucial to runtime operation. All of which
significantly increases the possibility of a problem materializing creating
problems during operation. 

\section{Evaluation of external libraries}
\paragraph{MySQL}
Being able to access and edit the database is very dangerous in a system that
contains such sensitive data as patient journals.
As noted, the EDB-ID 40678 vulnerability in MySQL could be used to gain root
access to the database. It requires local access, but since OpenMRS can be used
for example in a hospital with a lot of local users of different privileges (for
example Medical Student, Laboratory Assistants and of course Doctors), the
vulnerability is very relevant to OpenMRS. 

\paragraph{Tomcat}
Since a lot of servers running OpenMRS are Linux servers running e.g. Ubuntu
(the official installation instructions list Windows and Ubuntu installation
steps), EDB-ID 40450 is a dangerous exploit. It can be exploited through a web
shell delivered through for example cross-site scripting or SQL injection, and
can give root access to the server, giving the attacker full control.

\section{Recommendations}

\subsection{Recommendations for developers}

\paragraph{OWASP Top 10}
Of the errors found by Fortify, most were in the current (2013) OWASP Top 10
category of Security Misconfiguration, which includes default accounts that are
enabled and unmodified. As previously stated, there were examples of default
administrator users being asked to be removed by the system administrator. The
safer approach would be to force the sysadmin to create a new
administrator-level user and then automatically delete the default one. The same
goes for database access credentials, and maybe also encryption keys, especially
if the default enctryption method is AES. These vulnerabilities are, according to OWASP,
easy to exploit, but also quite easy to find, and would therefore be a good
target for something the developers could improve on, even just by providing
documentation to system administrators of how to perform scans checking whether
default values are in use.

OpenMRS, using default SSL certificates for the system, could also be considered
as part of the OWASP category Broken Authentication and Session Management,
which has become the second most common vulnerability on the list. According to
OWASP, these types of vulnerabilities can have severe consequences, letting an
attacker gain control over any user account, and possibly even all user accounts.
These errors are more difficult to exploit than the Security Misconfiguration
ones, but also more difficult to find. The OWASP recommendations for this
category of vulnerabilities is more diffuse, especially for the specific issue
of SSL certificates. The entry for Broken Authentication and Session management
also links to the category Sensitive Data Exposure.

Sensitive Data Exposure considers cases of data being transmitted in a way which
can be stolen. Fortify lists just over 60 privacy violations that are either critical
or high on the vulnerability spectrum, and some of these are found in such
classes as HttpClient, which handles HTTP POST data, and which doesn't encrypt sent
data. There are also multiple places the application logs things such as the
username performing an operation. While this can help find the identity of
an attacker, it also lets an attacker find information about the registered
users, easing for example bruteforce attacks.
According to OWASP, Sensitive Data Exposure should be mitigated by encrypting
all sensitive data that's sent, as well as not storing sensitive data
unnecessarily

ZAP found one SQL Injection vulnerability. Injection vulnerabilities are the
most common category on OWASP, and have been so for a long time. SQL injection
attacks are easy to perform and can have a severe impact. In this case, the
vulnerability was found in the Forgotten Password page, which is accessible to
non-logged in users, and is therefore especially dangerous. As OWASP recommends
for mitigating injection attacks, there should be server side evaluation of form
input and all special characters should be escaped on the server \autocite[]{owasptop10}.


\paragraph{Race conditions}
Since race conditions are becoming an increasingly
more commmon cause of errors and can cause such errors as returning critical
data to non-authorized users, these errors are a potentially important area for
the developers to improve. It seems that the java library for concurrency is the
one used in the codebase. So as the ThreadSafe report stated there seems to be
a collection of potential bugs aggregating around handling of the modules and
there might also according to the Fortify rapport be some misconception around
how the singletons operate and when they are used. Then a possible change to take
into consideration is to implement the java concurrency utilities library in the
OpenMRS-Web library and especially around the handling of modules and filters. 


\subsection{Recommendations for users and implementers}
\paragraph{Recommended software}
The install instructions for OpenMRS links the user to Tomcat 6, and with a
generic statement about how ``there are issues with Tomcat 8.''. In light of the
Tomcat exploit found on Exploit-DB, the user should not be using a version of Tomcat
from the package manager of the operating system if running a Linux server.
Downloading the newest version of Tomcat 7 from Apache's homepage is therefore
the reccomended course of action.

\paragraph{User management}
OpenMRS has three unchangeable user groups, Anonymous, Authenticated and System
Developer. Floss Manuals (which the official OpenMRS wiki refers to), there are
some specific recommendations for the privileges these users should recieve.
From FlossManuals: ``Anonymous privileges should be kept very restricted, since patient
information might otherwise be compromised. Privileges granted to the
Authenticated role are granted to anyone that logs into your system, no matter
what other role(s) they might be assigned. Granting privileges to the
Authenticated role is an easy way to grant privileges to all users of the
system. The System Developer role is automatically granted full access to the
system and should only be granted to system administrators.

Super users (system administrators) are automatically granted all privileges
in the system; therefore, you must be very careful to protect your system
administrator password.'' \autocite[]{FLOSS}
We recommend that users of the application follow these guidelines, as not
complying can lead to granting privileges to all users of the system, ergo more
easily letting attackers gain information not usually available to them.

Also, as noted earlier, system administrators should take extra care to remove
default users such as the user ``admin'' with the password ``Admin123''.

\section{Summary}

The static analysis was done with the tools Fortify, FindBugs and ThreadSafe.
All of the tools were easy to set up and the initial scans were easy to initiate,
except Fortify which required a bit of tinkering to allocate enough memory for it. All of the
tools have more advanced features and the possibility for tailoring the settings
more appropriate the needs of the project, both of which we did not have the
time or experience to do. 

A note on FindBugs: During our project the maintainer of FindBugs discontinued
it. This did not affect our project, but an alternative for future projects
might be SonarQube, which is another static analysis tool that seems more
user-friendly and with more complete results(bigger rule collection).

The dynamic analysis tool we used was OWASP Zap. This program has a steeper
learning curve and the results were a lot harder to analyze. For a more
effective use than what we achieved there is a need for a better understanding
of functionalities the program offers at the same time to have a good
understanding of the results there is a certain amount of understanding of
the program need.

Even with our short time using the tools the effort did return interesting
results in all areas of the codebase giving rapports of potential errors of
very different nature, for example concurrency vulnerabilities and weakness in
design of users and roles. Especially the concurrency vulnerabilites seem
prevalent. Since OpenMRS is a Java application and concurrency is not a first-class 
citizen in Java it is essential to have a good tool for finding errors in the
implementation of concurrent classes, as the bugs found by ThreadSafe confirms.

The potential weaknesses we found in the design of roles and users could not be
discovered with static analysis alone. We recognize that automatic tools for
analysis is not enough, so to discover faults in the design manual analysis and
learning how the code is implemented is necessary.

To finish off, we would like to state that since this is our first foray into
security testing, the results should be taken with a grain of salt, and might
not reflect the real state of security in OpenMRS. However, we feel that we
found relevant issues in many areas, and consider the analysis a success.

\end{document}


