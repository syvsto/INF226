
\documentclass{report} %skal vi legge til [a4paper] og endre til report? 

\usepackage[a4paper]{geometry}
\usepackage[backend=bibtex, style=verbose-trad2]{biblatex}
\usepackage[utf8]{inputenc}
\usepackage{comment}
\bibliography{kilder}

\title{INF226 Obligatory assignment}
\date{2016-10-02}
\author{Håvar Eggereide and Syver Storm-Furru}

\begin{document}

\subsection{ThreadSafe} % intro

``ThreadSafe is a static analysis tool for finding concurrency bugs and
potential performance issues in Java programs.''\autocite[]{ThreadSafe}

The program is developed by Contemplate ltd. aiming to help developers discover
concurrency issues in the code-base prior to the deploying of the application.
Concurrency is a complicated issue, simply having the correct understanding of
what it is and the theory of what it does isn't for everyone. Then the next step
of implementing it correctly can give rise to more complicated issues which are
even harder to discover. For the developer to have a tool to aid in working with
concurrency is of high value. Should a issue arise during runtime the
ramifications, for a system like OpenMRS which potentially need to be operating
24 hours a day, could possibly be disruption of the integrity of the database
and halt operations.

ThreadSafe is not open source and it is necessary to invest in a license if
there is a need to integrate the application in the SDLC. There is however a
free trial of 14 days available.

\paragraph{ThreadSafe} % Setup and Configuration

ThreadSafe has three install options, two which are available in the trial
version. These options are: A .jar file which is executable from the terminal
taking input-arguments and a setup file for each project, plugin for a static
analyzing tool called SonarQube and a plugin for the Eclipse IDE.

As we were using a free trial we had to be limited to ether using the terminal
program or the Eclipse plugin, which were the included options in the trial. As
setting up a project configuration file for the terminal application seemed to
open the possibility of misconfiguration, using the Eclipse plugin seemed the
best alternative. Installing the plugin from the .zip file was easy, the only
other thing necessary was to order a trial license and past this license in the
Preferences -> ThreadSafe. We then let Eclipse build the Maven .pom file from
the OpenMRS source-code and then everything was ready for scanning. 

\paragraph{ThreadSafe} % Test of the software tools

The initial scan gave a few possible errors, all of them concerning ether the
OpenMRS-API or the OpenMRS-WEB which might make sense as these expose surface to
external modules. The scan also gave some false positives referring to
unit-test.

\paragraph{ThreadSafe} % Raw data

Excluding false positives from tests the scan gave 12 major and 1 minor
warnings. No Critical or ``Blocker'' warnings were generated. The major warnings
are as follows:

\begin{itemize}
\item Inconsistent synchronization of access to Field/a collection
\item Mixed synchronization of access to a field
\item No lock held while iterating on a synchronized collection view
\item Non atomic use of Get/Check/Put
\item Synchronizing on reusable objects
\item Unsafe iteration over synchronized collection
\item Unsynchronized access to field from asynchronously invoked method
  
 

%Interpretations of the raw data

\paragraph{Race conditions}


%Recommandations
\paragraph{Race conditions}


%  LocalWords:  SonarQube

\end{document}
%  LocalWords:  runtime SDLC ThreadSafe OpenMRS
