\message{ !name(innlevering.tex)}\documentclass{article}

\usepackage[backend=bibtex, style=verbose-trad2]{biblatex}
\bibliography{kilder}

\title{INF226 Obligatory assignment}
\date{2016-10-02}
\author{Håvar Eggereide and Syver Storm-Furru}

\begin{document}

\message{ !name(innlevering.tex) !offset(-3) }

\maketitle
\pagenumbering{gobble}
\newpage
\pagenumbering{arabic}
\tableofcontents
\newpage
\section{Introduction}

  For our obligatory assignment we were tasked with analyzing OpenMRS, a medical
  patient journal system.

\subsection{OpenMRS}

  OpenMRS is, according to the website, the ``world's leading open source
  enterprice electronic medical record system platform'' \autocite[]{OpenMRSAbout}. It is used in
  hospitals all over the world, for example in Nigeria, South Africa, India
  and the United States, and is supported by many different governments, NGOs,
  and both for- and non-profit organizations. The software has a stated goal of
  being uasble with no programmming knowledge, and to be a common platform for
  which medical informatics efforts in developing can be built.

  \subsubsection{Technical Specifications}

  OpenMRS is a client-server platform, with a web frontend. It is programmed in
  Java 7, using Tomcat 6 or 7 as the server framework, and MySQL 5.6 as the
  database backend. It also exposes a programming API to users, and is modular
  and extendable.

\subsubsection{Setup}

  The setup process of OpenMRS is quite involved and time consuming when
  attempting to do so on a personal computer, requiring both Tomcat, Java and
  MySQL to be setup. The official documentation is useful, but different parts
  of it uses different versions of e.g. Tomcat, so it can be confusing.
  It also provides install instructions for Windows and Linux distributions
  with Aptitude, but not for OSX or other Linux distributions. 

  As mentioned, OpenMRS doesn't run on the newest version of MySQL (at the time
  of writing MySQL 5.7), and the install instructions do not mention this. The
  process of figuring this out, and of removing and reinstalling a previous
  version of MySQL, proved to be a lengthy detour on an already long road.
  The instructions are also not very specific when noting which files you need
  to run OpenMRS in Tomcat, whether it is the source code, which was difficult
  to build and only return a test suite on a normal compile, a readily packaged
  complete install (which did not work properly), or a .war file that should be
  uploaded to the Tomcat server.


  We first attempted setting up OpenMRS on OSX 10.11, but ran into problems when
  trying to install the correct version of MySQL, and therefore retried in an
  empty virtual machine running Linux (tested with both Lubuntu and Kali Linux).
  Following the install instructions were a lot easier when running Debian based
  distributions containing Aptitude, but we still had to find and install a
  previous version of MySQL. 


  Once everything was installed OpenMRS had some extra setup that was required,
  done through a web interface. This was mostly easy once the correct version of
  MySQL was in place. You were also given a first username and password that
  was, respectively, `admin' and `Admin123'.

\subsubsection{Usage}


\subsection{HPE Fortify}

  Fortify is a static code analysis tool suite, developed by Hewlett-Packard
  Enterprise. It aims to ``make application security a natural part of the new
  SDLC, enabling time to market by building security in''\autocite[]{Fortify}.

  Fortify is a propietary solution, but is available with an academic license
  for free.

\subsubsection{Audit Workbench}

  

\subsection{OWASP ZAP}
\subsection{FindBugs}


\end{document}
\message{ !name(innlevering.tex) !offset(-96) }
